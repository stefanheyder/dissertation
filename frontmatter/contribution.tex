
\pdfbookmark[1]{Publications and Contributions}{publications}
\chapter*{Publications and Contributions}
\begin{refsection}[ownpubs]
This thesis consists mostly of unpublished work. 
During my time as a PhD student, I have, however, been fortunate to collaborate with many scientists on problems in mathematical epidemiology with a focus on COVID-19, resulting in several preprints and publications.
In this section I aim to clarify my contributions to these works were and distinguish them from the contributions of the present thesis. Peer-reviewed publications are marked with a superscript$^{\mathparagraph}$.

The first set of works follows the convention of sorting authors by their last name.


\paragraph{\cite{Hotz2020Monitoring}}
Thomas Hotz conceived the initial idea for this paper, derived most of the results, and wrote the initial manuscript. Matthias Glock and I managed the data cleaning, estimation of reproduction numbers and automation of the associated dashboard. Alexander Krämer, Anne Böhle, and Sebastian Semper provided consultation on epidemiological and practical relevance. 

\paragraph{\cite{Burgard2021Regional}$^{\mathparagraph}$}
Thomas Hotz initially suggested applying small area estimation techniques to reproduction number estimates. I developed the model, estimator and simulation study, consulting with Thomas Hotz throughout the process. I wrote the first draft of the paper. Jan Pablo Burgard provided consultations on the small-area aspect of the paper, and Tyll Krüger on the epidemiological implications.

\paragraph{\cite{Grundel2022How}$^{\mathparagraph}$}
Sarah Grundel and Karl Worthmann developed the idea of using optimal control techniques to balance testing with non-pharmaceutical interventions under societal constraints.
Thomas Hotz and I designed the compartmental model, incorporating realistic parameters, and derived the epidemiological implications of the optimal strategies. 
Tobias Ritschel and Philipp Sauerteig established the mathematical and numerical results related to optimal control theory.
The writing for the initial version of the paper was divided among Tobias Ritschel, Philipp Sauerteig and I. 

\paragraph{\cite{Grundel2021How}$^{\mathparagraph}$}
Sarah Grundel and Karl Worthmann came up with the idea of applying the optimal control techniques from testing to vaccination strategies. Thomas Hotz and myself consulted on how to adapt the compartmental model to account for vaccination instead of testing and contributed to the epidemiological interpretation of the results. The initial version of the paper was written by Tobias Ritschel and Philipp Sauerteig.

\paragraph{\cite{Heyder2023Measures}}
Thomas Hotz and I conceived with the idea of comparing different measures of COVID-19 spread with respect to ease of interpretation and communication. I then developed these ideas into an initial manuscript, except for the derivation of the renewal equation, which was contributed by Thomas Hotz. \\[18pt]
The second set of works follows the convention of sorting authors by contribution.

\paragraph{\cite{Bracher2021Preregistered}$^\mathparagraph$}
\paragraph{\cite{Bracher2022National}$^\mathparagraph$}
\paragraph{\cite{Sherratt2022Predictive}$^\mathparagraph$}
\paragraph{\cite{Brockhaus2023Why}$^\mathparagraph$}
\paragraph{\cite{Wolffram2023Collaborative}$^\mathparagraph$}

\printbibliography[heading=none]
\end{refsection}
