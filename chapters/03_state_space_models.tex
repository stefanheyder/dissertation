\chapter{State space models}
\label{cha:state_space_models}

State space models are a versatile class of statistical models which allow to model non-stationary time series data and come along with straight-forward interpretation.
The main idea of these models is to introduce unobserved \textbf{latent states} whose joint distribution is given by a Markov process and model the observed time series conditional on theses states.
By exploiting this structure, inference in state space models becomes computationally efficient, i.e. the complexity of algorithms is linear with respect to the number $n$ of time points considered.

An additional advantage, that will become more explicit in \Cref{sec:modelling_epidemiological_dessiderata_with_state_space_models}, is that state space models allow to interpret the modeled dynamics of latent states which makes 


\begin{definition}[State Space Model]
    \label{def:ssm}
    A \textbf{state space model} is a discrete time stochastic process $(X_t, Y_t)_{t=0, \dots, n -1}$ taking values in a measurable space $\mathcal X \times \mathcal Y$ such that
    \begin{enumerate}
        \item The marginal distribution of the \textbf{states} $(X_0, \dots, X_{n - 1})$ is a discrete time Markov process, i.e. for $t = 1, \dots, n-1$ 
        \begin{align}
            \label{eq:markov_property}
            \P \left( X_{t} \in B \middle| X_0, \dots, X_{t - 1} \right) = \P \left( X_{t} \in B \middle| X_{t - 1} \right)
        \end{align}
        for all measurable $B \subseteq \mathcal X$. 
        \item Conditional on the state $X_t$ and observation $Y_{t - 1}$, $Y_t$ is independent of $X_s$ and $Y_{s - 1}$, $s < t$, i.e. 
        \begin{align*}
            \P \left( Y_{t} \in B \middle| X_{0}, \dots, X_{t}, Y_{0}, \dots, Y_{t - 1} \right) &= \P \left( Y_{t} \in B | X_{t}, Y_{t - 1} \right)
        \end{align*}
        for all measurable $B \subseteq \mathcal{Y}$.
    \end{enumerate}
\end{definition}

For notational convenience we will write $X_{s:t} = \left(X_s, \dots, X_{t}\right)$ for the vector that contains all states from $s$ to $t$, dropping the index if we consider the whole set of observations, so $X = X_{0:n-1}$ 
Similarly we set $Y_{s:t} = \left(Y_s, \dots, Y_{t}\right)$ and $Y = Y_{0:n-1}$.

\todo{picutre of dependency structure}

\begin{remark}
    Contrary to the standard definition of a state space model, our \Cref{def:ssm} allows $Y_t$ to depend on $Y_{t - 1}$.
    This is not a limitation of the standard definition: given a state space model of the form in \Cref{def:ssm} we can transform it to the standard form by choosing states $(X_t, Y_t) \in \mathcal X \times \mathcal Y$ and observations $Y_t \in \mathcal Y$ such that the state space model becomes a stochastic process on $ \left( \mathcal X \times \mathcal Y\right) \times \mathcal Y$.

    Additionally most computations and inferences in this thesis will condition on the observations $Y$ and as such $Y$ may be treated as fixed. The only exception to this is in simulation studies where we sample from the joint distribution of $(X, Y)$.
    
    As the models considered in \Cref{cha:analysis_of_selected_models} will make extensive use of state space models with this dependency structure we opt to use this non-standard definition here.

    In most models we consider in this thesis we use $\mathcal X = \R^m$, $\mathcal Y = \R^p$ or $\mathcal Y = \Z^p$ so that $\mathcal X$ is $m$ dimensional and $\mathcal Y$ is $p$ dimensional and equip these spaces with the usual Borel $\sigma$-Algebras.
\end{remark}

Most models that I consider in this thesis will admit densities for the state transitions w.r.t. a common dominating measure $\mu_{\mathcal X}$ and similar for the observations w.r.t. a (potentially different) domination measure $\mu_{\mathcal Y}$. \todo{check whether there are models that violate this} 

\begin{notation}[Densities, conditional densities]
    I will use the standard abuse of notation for densities that makes the type of density \glqq{}obvious\grqq{} from the arguments used.
    This means that $p(x)$ is the density for all states $X$, $p(x_t|x_{t - 1})$ the conditional density of $X_t|X_{t - 1}$ and similarly for observations: $p(y|x)$ is the density of all observations $Y$ conditional on all states $X$.

    Note that this notation also implicitly includes the time $t$ and allows for changes in, \eg, the state transition over time.

    When densities stem from a parametric model parametrized by $\theta \in \Theta \subseteq \mathbf{R}^{k}$ and the dependence of the model on $\theta$ is of interest, i.e. because we try to estimate $\theta$, we indicate this by adding a subscript to the densities.
    If the dependence is not of interest, e.g. because $\theta$ is fixed, I will usually omit $\theta$ for better readability.

    In this notation the joint density of a parametric state space model factorizes as
    \begin{align*}
        p_\theta(x,y) &= p_\theta(x_0, \dots, x_{n - 1}, y_0, \dots, y_{n - 1}) \\
        &= p_\theta (x_0)\prod_{t = 1}^{n - 1} p_\theta(x_{t}|x_{t - 1}) \prod_{t = 0}^{n - 1} p_\theta(y_t | x_t, y_{t - 1}),
    \end{align*}
    where $p_\theta(y_0|x_0, y_{-1}) = p_\theta(y_0, x_0)$. 

\end{notation}

Given data $(y_t)_{t = 0, \dots, n - 1}$ that may be modeled with a state space model the practitioner is confronted with several tasks, which provides the structure of this chapter:

\begin{enumerate}
    \item\label{it:model_choice} Choosing a suitable, usually parametric, class of state space models that include the effects of interest.
    \item\label{it:model_fitting} Fitting such a parametric model to the data at hand by either frequentist or bayesian techniques.
    \item\label{it:smoothing_problem} Infer about the latent states $X$ from the observations $Y$ by determining, either analytically or through simulation, the smoothing distribution $X|Y$.
\end{enumerate}

The first step, \cref{it:model_choice}, requires that the practitioner specifies a joint probability distribution for the states and observations (\Cref{sec:modelling_epidemiological_dessiderata_with_state_space_models}).
Due to the assumed dependency structure this boils down to specifying transition kernels for the states and observations.
The setting \Cref{def:ssm} is too abstract to perform inference in so further assumptions on the types of distributions for the latent states and observations are needed. 
In this chapter we will discuss linear gaussian state space models (\Cref{sec:linear_gaussian_state_space_models}), where both the posterior distribution and the likelihood are analytically available. For the epidemiological application we have in mind these are however insufficient due to the non-linear behaviour of incidences and the low count per region (\Cref{sec:dessiderata}).
Such observations are better modeled with distributions on the natural numbers, i.e. with a Poisson or negative binomial distribution, leading to the class of logconcave Gaussian state space models (\Cref{sec:logconcave_gaussian_state_space_models}).

Regarding the second step, \cref{it:model_fitting}, a frequentist practitioner will want to perform maximum likelihood inference on $\theta$.
While asymptotic confidence intervals for $\theta$ can be derived both theoretically and practically \cite[Chapter 7]{Durbin2012Time}, they are, in the context of this thesis, usually of little interest.
We choose to view this fitting as an Empirical Bayes procedure and our main practial interest lies in analyzing the posterior distribution $X|Y$.  

To obtain the maximum likelihood estimates $\hat\theta$ one needs access to the likelihood 
\begin{align}
    \label{eq:likelihood}
    p(y) = \int_{\mathcal X^n} p(x,y) \d x,
\end{align}
which is usually not analytically available.
Direct numerical evaluation of \Cref{eq:likelihood} is hopeless due to the high dimensionality of the state space $\mathcal X^n$. 
Instead we will resort to simulation based inference by importance sampling (see \Cref{sec:importance_sampling}), an alternative would be particle filters \cite{Chopin2020Introduction}.

The performance of these simulations depends crucially on constructing distributions that are close to the posterior $p(x|y)$ but are easy to sample from. To this end we construct suitable Gaussian state space models (\Cref{sec:gaussian_importance_sampling_for_state_space_models}) in which sampling from the posterior is analytically possible. 
This will be a good strategy if the target posterior $p(x|y)$ can be well approximated by a Gaussian distribution --- otherwise we may want to account for multiple modes by considering mixtures of Gaussian state space models or account for heavy tails with t-distributed errors (\Cref{sec:accouting_for_multimodality_and_heavy_tails}). 

\section{Modelling epidemiological dessiderata with state space models}
\label{sec:modelling_epidemiological_dessiderata_with_state_space_models}

\section{Linear Gaussian state space models}
\label{sec:linear_gaussian_state_space_models}

\begin{itemize}
    \item joint model is gaussian
    \item filtering distribution obtained by Kalman filter
    \item smoothing distribution obtained by Kalman smoother
    \item variants: sqrt filter / precision filter
    \item gaussian likelihood analytically available, MLE can be found by numerical methods (gradient descent or EM, depending on problem)
    \item computation is efficient: linear in time dimension $n$
    \item $Y_{t + 1}$ may also depend on $Y_{t}$ as we will target the conditional distribution anyways
\end{itemize}

\section{Logconcave Gaussian state space models}
\label{sec:logconcave_gaussian_state_space_models}

\begin{itemize}
    \item replace gaussian observations with log concave observations
    \item motivation for logconcave distributions: posterior has unique mode, because up to constants $\log p(x | y) = - \frac{1}{2} (x - \mu)^T \Sigma^{-1} (x - \mu) + \log p(y | x) $ so $\log p(x|y)$ is concave
    \item not restricted to same type of distribution per time step (though in ISSSM it will be)
    \item Laplace approximation sensible for these types of models: single mode
    \item special case: exponential family distributions
\end{itemize}

\section{Importance Sampling}
\label{sec:importance_sampling}

Suppose we have a function $h: \mathcal X \to \R$ whose integral $$\zeta = \int_{\mathcal X} h(x) \d x$$ we want to compute. 
Furthermore suppose that we can write 
$$
    \int_{\mathcal X} h(x) \d x = \int_{\mathcal X} f(x) \d \P(x)
$$
for a probability measure $\P$ and function $f: \mathcal X \to \R$. 
Let $\G$ be another measure on $\mathcal X$ such that $f\P$ is absolutely continuous with respect to $\G$, $f\P \ll \G$ and let $v = \frac{\d f\P}{\d\G}$ be the corresponding Radon-Nikodym derivative. Then 
$$
\zeta = \int_{\mathcal X} h(x) \d x = \int_{\mathcal X} f(x) \d \P(x) = \int_{\mathcal X} v(x)\d\G(x)
$$
which suggests to estimate $\zeta$ by Monte-Carlo integration: $$\hat \zeta = \frac 1 N \sum_{i=1}^{N} v(X_i)$$ for $X_i \iid \G$, $i = 1, \dots, N$. 

If one is not interested in a particular $h$ but rather in an approximation of $\P$ and $\P$ is absolutely continuous with respect to $\G$, then one may view $$\hat \P_N = \frac{1}{N} \sum_{i = 1}^{N} v(X_i) \delta_{X_i}$$ as a particle approximation of $\P$. In this setting \cite{Agapiou2017Importance} shows that the random measure $\hat \P_N$ converges to $\P$ at rate $\mathcal O\left(\frac 1 N\right)$\todo{check} in an appropriate metric. 

To perform importance sampling one must be able to evaluate the weights $v$. In a bayesian setting this is usually infeasible: if $\P$ is a posterior then the integration constant of its density is intractable. 
In this case one can usually evaluate the weights up to a constant, i.e. $w(x) \propto_x \frac{\d \P}{\d \G}(x)$ is available. The missing constant is then $\int w(x) \d \G$ which is itself amenable to importance sampling.

This leads to the self-normalized importance sampling weights $W_i = \frac{w(X_i)}{\sum_{i = 1}^N w(X_i)}$ and Monte Carlo estimates $\hat \zeta = \sum_{i = 1}^{N} W_i f(X_i)$ and particle approximation $\hat \P_N = \sum_{i = 1}^{N} W_i \delta_{X_i}$. 

In both cases one can show that once second moments of $w$ with respect to $\G$ exist the Monte-Carlo estimates are consistent and asymptotically normal at the usual rates, see \cite[Chapter 8]{Chopin2020Introduction}. 

Importance sampling is useful in situations where simulation from $\P$ is not feasible or when Monte Carlo integration with respect to $\P$ is unattractive due to high variance estimates. 

\begin{itemize}
    \item importance sampling as a variance reduction technique
    \item importance sampling as a technique to make intractable distributions tractable
    \item importance sampling vs. other methods:
        \begin{itemize}
            \item vs. ABC
            \item vs. MCMC
            \item vs. INLA (isn't this MCMC?)
        \end{itemize}
    \item measuring how good IS performs: ESS and other measures
    \item related results regarding performance of IS (Chatterje, Agapiou)
\end{itemize}

\subsection{Laplace approximation}

\begin{itemize}
    \item approximate at mode, problematic if posterior is not unimodal (but then gaussian approximation probably not worth it)
\end{itemize}
\subsection{Cross entropy method}
\begin{itemize}
    \item 
\end{itemize}
\subsection{Efficient importance sampling}


\section{Gaussian importance sampling for state space models}
\label{sec:gaussian_importance_sampling_for_state_space_models}

As the likelihood of a general state space model is neither analytically nor numerically tractable one has to resort to Monte-Carlo techniques.
Recall that the likelihood is a high-dimensional integral of the form
\begin{align*}
    \lik(\theta) = p_\theta(y) = \int p_\theta(y,x) \d x = \int p_\theta(y|x) p_\theta(x) \d x = \E p_\theta(y|X).
\end{align*}
By the standard law of large numbers we can approximate $\lik(\theta)$ by 
\begin{align*}
  \hat\lik(\theta) = \frac 1 N \sum_{i=1}^N p_\theta(y|X^i)  
\end{align*}
for $N\in\N$ samples $X^i \iid p(x)$. 
However, the variance of $\hat\lik(\theta)$ is likely to be very high if samples $X^i$ are drawn from the prior distribution $p(x)$ as they are not informed by the observations $y$. 
As $p_\theta(x|y) \propto p_\theta(x,y)$ a more promising approach would be to use samples $X^i \sim p_\theta(x|y)$, but this distribution is usually not available. 

While bayesian computational approaches such as MCMC\todo{cite} are able to generate (approximate) samples from this posterior distribution, importance sampling tries to find a distribution close to the target and re-weighs samples to ensure unbiased estimates of $\lik(\theta)$.

\section{Accouting for multimodality and heavy tails}
\label{sec:accouting_for_multimodality_and_heavy_tails}


\section{Maximum likelihood estimation}
\label{sec:maximum_likelihood_estimation}



%\section{State space models in high dimensions}
%\label{sec:state_space_models_in_high_dimensions}
