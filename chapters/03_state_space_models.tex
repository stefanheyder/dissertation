\glsresetall
\chapter{State space models}
\label{cha:state_space_models}

State space models are a versatile class of statistical models which allow to model non-stationary time series data and come along with straight-forward interpretation.
The main idea of these models is to introduce unobserved \textbf{latent states} whose joint distribution is given by a Markov process and model the observed time series conditional on theses states.
By exploiting this structure, inference in \glspl{ssm} becomes computationally efficient, i.e. the complexity of algorithms is linear with respect to the number $n$ of time points considered.

An additional advantage, that will become more explicit in \Cref{sec:modelling_epidemiological_dessiderata_with_state_space_models}, is that \glspl{ssm} allow to interpret the modeled dynamics of latent states which makes


\begin{definition}[State Space Model]
    \label{def:ssm}
    A \textbf{\gls{ssm}} is a discrete time stochastic process $(X_t, Y_t)_{t=0, \dots, n -1}$ taking values in a measurable space $\mathcal X \times \mathcal Y$ such that
    \begin{enumerate}
        \item The marginal distribution of the \textbf{states} $(X_0, \dots, X_{n - 1})$ is a discrete time Markov process, i.e. for $t = 1, \dots, n-1$
              \begin{align}
                  \label{eq:markov_property}
                  \P \left( X_{t} \in B \middle| X_0, \dots, X_{t - 1} \right) = \P \left( X_{t} \in B \middle| X_{t - 1} \right)
              \end{align}
              for all measurable $B \subseteq \mathcal X$.
        \item Conditional on the state $X_t$ and observation $Y_{t - 1}$, $Y_t$ is independent of $X_s$ and $Y_{s - 1}$, $s < t$, i.e.
              \begin{align*}
                  \P \left( Y_{t} \in B \middle| X_{0}, \dots, X_{t}, Y_{0}, \dots, Y_{t - 1} \right) & = \P \left( Y_{t} \in B | X_{t}, Y_{t - 1} \right)
              \end{align*}
              for all measurable $B \subseteq \mathcal{Y}$.
    \end{enumerate}
\end{definition}

For notational convenience we will write $X_{s:t} = \left(X_s, \dots, X_{t}\right)$ for the vector that contains all states from $s$ to $t$, dropping the index if we consider the whole set of observations, so $X = X_{0:n-1}$
Similarly we set $Y_{s:t} = \left(Y_s, \dots, Y_{t}\right)$ and $Y = Y_{0:n-1}$.

\todo{picutre of dependency structure}

\begin{remark}
    Contrary to the standard definition of a \gls{ssm}, our \Cref{def:ssm} allows $Y_t$ to depend on $Y_{t - 1}$.
    This is not a limitation of the standard definition: given a \gls{ssm} of the form in \Cref{def:ssm} we can transform it to the standard form by choosing states $(X_t, Y_t) \in \mathcal X \times \mathcal Y$ and observations $Y_t \in \mathcal Y$ such that the \gls{ssm} becomes a stochastic process on $ \left( \mathcal X \times \mathcal Y\right) \times \mathcal Y$.

    Additionally, most computations and inferences in this thesis will be conditioned on a single set of observations $Y$. As such $Y$ may be treated as fixed and $Y_{t}$ only depends on $X_{t}$. The only exception to this is in simulation studies where we sample from the joint distribution of $(X, Y)$.

    As the models considered in \Cref{cha:analysis_of_selected_models} will make extensive use of \glspl{ssm} with this dependency structure we opt to use this non-standard definition here.

    In most models we consider in this thesis we use $\mathcal X = \R^m$, $\mathcal Y = \R^p$ or $\mathcal Y = \Z^p$ so that $\mathcal X$ is $m$ dimensional and $\mathcal Y$ is $p$ dimensional and equip these spaces with the usual Borel $\sigma$-Algebras.
\end{remark}

Most models that I consider in this thesis will admit densities for the state transitions w.r.t. a common dominating measure $\mu_{\mathcal X}$ and similar for the observations w.r.t. a (potentially different) domination measure $\mu_{\mathcal Y}$. \todo{check whether there are models that violate this}

\begin{notation}[Densities, conditional densities]
    I will use the standard abuse of notation for densities that makes the type of density \glqq{}obvious\grqq{} from the arguments used.
    This means that $p(x)$ is the density for all states $X$, $p(x_t|x_{t - 1})$ the conditional density of $X_t|X_{t - 1}$ and similarly for observations: $p(y|x)$ is the density of all observations $Y$ conditional on all states $X$.

    Note that this notation also implicitly includes the time $t$ and allows for changes in, \eg, the state transition over time.

    When densities stem from a parametric model parametrized by $\theta \in \Theta \subseteq \mathbf{R}^{k}$ and the dependence of the model on $\theta$ is of interest, i.e. because we try to estimate $\theta$, we indicate this by adding a subscript to the densities.
    If the dependence is not of interest, e.g. because $\theta$ is fixed, I will usually omit $\theta$ for better readability.

    In this notation, the joint density of a parametric \gls{ssm} factorizes as
    \begin{align*}
        p_\theta(x,y) & = p_\theta(x_0, \dots, x_{n - 1}, y_0, \dots, y_{n - 1})                                                              \\
                      & = p_\theta (x_0)\prod_{t = 1}^{n - 1} p_\theta(x_{t}|x_{t - 1}) \prod_{t = 0}^{n - 1} p_\theta(y_t | x_t, y_{t - 1}),
    \end{align*}
    where $p_\theta(y_0|x_0, y_{-1}) = p_\theta(y_0, x_0)$.

    As inferences we make in this thesis depend on the \gls{ssm} only through the likelihood we identify almost sure versions of $(X, Y)$ with itself, i.e. all equations involving $X$ or $Y$ are understood almost surely.
\end{notation}

Given data $(y_t)_{t = 0, \dots, n - 1}$ that may be modeled with a \gls{ssm} the practitioner is confronted with several tasks, which provide the structure of this chapter:

\begin{enumerate}
    \item\label{it:model_choice} Choosing a suitable, usually parametric, class of \glspl{ssm} that include the effects of interest.
    \item\label{it:model_fitting} Fitting such a parametric model to the data at hand by either frequentist or Bayesian techniques.
    \item\label{it:smoothing_problem} Infer about the latent states $X$ from the observations $Y$ by determining, either analytically or through simulation, the smoothing distribution $X|Y$.
\end{enumerate}

The first step, \cref{it:model_choice}, requires that the practitioner specifies a joint probability distribution for the states and observations (\Cref{sec:modelling_epidemiological_dessiderata_with_state_space_models}).
Due to the assumed dependency structure this boils down to specifying transition kernels for the states and observations.
The setting \Cref{def:ssm} is too abstract to perform inference in so further assumptions on the types of distributions for the latent states and observations are needed.
In this chapter we will discuss \gls{glssm}  (\Cref{sec:linear_gaussian_state_space_models}), where both the posterior distribution and the likelihood are analytically available. For the epidemiological application we have in mind these are however insufficient due to the non-linear behaviour of incidences and the low count per region (\Cref{sec:dessiderata}).
Such observations are better modeled with distributions on the natural numbers, i.e. with a Poisson or negative binomial distribution, leading to the class of logconcave Gaussian state space models (\Cref{sec:logconcave_gaussian_state_space_models}).

Regarding the second step, \cref{it:model_fitting}, a frequentist practitioner will want to perform maximum likelihood inference on $\theta$.
While asymptotic confidence intervals for $\theta$ can be derived both theoretically and practically \cite[Chapter 7]{Durbin2012Time}, they are, in the context of this thesis, usually of little interest.
We choose to view this fitting as an Empirical Bayes procedure and our main practial interest lies in analyzing the posterior distribution $X|Y$.

To obtain the maximum likelihood estimates $\hat\theta$ one needs access to the likelihood
\begin{align}
    \label{eq:likelihood}
    p(y) = \int_{\mathcal X^n} p(x,y) \d x,
\end{align}
which is usually not analytically available.
Direct numerical evaluation of \Cref{eq:likelihood} is hopeless due to the high dimensionality of the state space $\mathcal X^n$.
Instead we will resort to simulation based inference by importance sampling (see \Cref{sec:importance_sampling}), an alternative would be particle filters \cite{Chopin2020Introduction}.

The performance of these simulations depends crucially on constructing distributions that are close to the posterior $p(x|y)$ but are easy to sample from. To this end, we construct suitable Gaussian state space models (\Cref{sec:gaussian_importance_sampling_for_state_space_models}) in which sampling from the posterior is analytically possible.
This will be a good strategy if the target posterior $p(x|y)$ can be well approximated by a Gaussian distribution --- otherwise, we may want to account for multiple modes by considering mixtures of Gaussian state space models or account for heavy tails with t-distributed errors (\Cref{sec:accouting_for_multimodality_and_heavy_tails}).

\section{Modelling epidemiological dessiderata with state space models}
\label{sec:modelling_epidemiological_dessiderata_with_state_space_models}

\section{Linear Gaussian state space models}
\label{sec:linear_gaussian_state_space_models}

\begin{itemize}
    \item joint model is gaussian
    \item filtering distribution obtained by Kalman filter
    \item smoothing distribution obtained by Kalman smoother
    \item variants: sqrt filter / precision filter
    \item gaussian likelihood analytically available, MLE can be found by numerical methods (gradient descent or EM, depending on problem)
    \item computation is efficient: linear in time dimension $n$
    \item $Y_{t + 1}$ may also depend on $Y_{t}$ as we will target the conditional distribution anyways
\end{itemize}

\gls{glssm} are the working horses of most methods used in this thesis because they are analytically tractable and computationally efficient. Indeed for fixed dimension of states $m$ and observations $p$ the runtime of algorithms that we consider in this thesis is $\mathcal O(n)$.

\begin{definition}[\gls{glssm}]
    \label{def:glssm}
    A \gls{glssm} is a state space model where states obey the transition equation
    \begin{align}
        \label{eq:glssm_states}
        X_{t + 1} & = A_{t}X_{t} + u_{t} \varepsilon_{t + 1} &  & t = 0, \dots, n - 1,
    \end{align}
    and observations obey the observation equation
    \begin{align}
        \label{eq:glssm_observations}
        Y_{t} & = B_{t}X_{t} + v_{t} + \eta_{t} &  & t = 0, \dots, n.
    \end{align}
    Here $A_{t} \in \mathbf{R}^{m \times m}$ and $B_{t} \in \mathbf{R}^{p \times m}$ are matrices the specify the systems dynamics. The \textbf{innovations} $\varepsilon_{t + 1}$ and \textbf{measurement noise} $\eta_{t}$ are independent from one another and from the starting value $X_{0} \sim \mathcal N (\E X_{0}, \Sigma_{0})$.

    Furthermore, $\varepsilon_{t+1} \sim \mathcal N(0, \Sigma_{t})$ and $\eta_{t}\sim \mathcal N(0, \Omega)$ are centered Gaussian random variables and $u_{t + 1}, t = 1, \dots, n-1$, $v_{t}, t = 0, \dots, n$ are deterministic biases.
\end{definition}

The defining feature of a \gls{glssm} is that its joint distribution is Gaussian.

\begin{lemma}
    A state space model can be written as a \gls{glssm} if and only if its joint distribution is Gaussian.
\end{lemma}
\begin{proof}
    \todo{technicailty: distringuish between a.s. version}
\end{proof}

As the joint distribution of $(X, Y)$ is Gaussian, so are conditional distributions of states given observations. Two such distributions are of interest: the \textbf{filtering distribution} is the conditional distribution of $X_{t}$ given all observations until time $t$, that is $Y_{0:t}$. When $t < n$ this is distinct from the \textbf{smoothing distribution}, i.e. the distribution of $X_{t}$ given all observations $Y$.
\todo{Note that the filtering distributions does not specify a valid joint distribution for the states, but the smoothing does.}

Both distributions may be obtained efficiently using the celebrated Kalman filter and smoother algorithms \todo{cite correclty}.

\begin{algorithm}
    \caption{Kalman filter}
    \KwIn{observations $y = (y_{0}, \dots, y_{n})$, \gls{glssm}}
    \KwOut{filtered expectations $\hat X_{t|t}$, covariance matrices $\Xi_{t|t}$, likelihood $p(y)$}
    \emph{Initialization}\;
    $\hat y_{0|-1} = B_{0}\hat x_{0|-1} + v_{t}$\;
    $\Psi_{0|-1} = B_{0} \Sigma_{0}B_{0}^{T} + \Omega_{0}$\;
    \emph{Prediction}\;
    \emph{Filter}\;
\end{algorithm}

\begin{algorithm}
    \caption{Kalman smoother}
    \KwIn{observations $y = (y_{0}, \dots, y_{n})$, \gls{glssm}}
    \KwOut{filtered expectations $\hat X_{t|t}$ and covariance matrices $\Xi_{t|t}$}
    \emph{Initialization}\;
    $\hat y_{0|-1} = B_{0}\hat x_{0|-1} + v_{t}$\;
    $\Psi_{0|-1} = B_{0} \Sigma_{0}B_{0}^{T} + \Omega_{0}$\;
    \emph{Prediction}\;
    \emph{Filter}\;
\end{algorithm}

Notice that the Kalman filter calculates the likelihood $p(y)$ while filtering --- this is possible because of the dependency structure of the state space model --- this makes inference via maximum likelihood possible in \gls{glssm}s.

To ensure numerical stability in these algorithms, the square root filter and smoother \cite{Morf1975Squareroot} may be used, see also \cite{Schneider1986Kalmanfilter} for an accessible introduction to it and other variants.

The Kalman smoother computes the marginal distributions $X_{t} | Y$ for $t = 0, \dots, n-1$ and, owing to the Markov structure of the states, these are enough to specify the joint distribution $X|Y$, allowing to simulate from it.

\begin{algorithm}
    \caption{Forwards filter, backwards smoother \cite[Proposition 1]{Fruhwirth-Schnatter1994Data}}
    \KwIn{TODO}
    \KwOut{TODO}
    \emph{Something}\;
\end{algorithm}

The modeling capacity of \gls{glssm} is, however, limited: most interesting phenomena follow neither linear dynamics nor are well modeled by a Gaussian distribution.
Nevertheless, linearization of non-linear dynamics suggests that  \gls{glssm}s may have some use as approximations to these more complicated phenomena, provided they are sufficiently close to Gaussian models, e.g. unimodal and without heavy tails.
We start to move away from linear Gaussian models by allowing observations that are non-Gaussian.

\section{Logconcave Gaussian state space models}
\label{sec:logconcave_gaussian_state_space_models}

\begin{itemize}
    \item replace gaussian observations with log concave observations
    \item motivation for logconcave distributions: posterior has unique mode, because up to constants $\log p(x | y) = - \frac{1}{2} (x - \mu)^T \Sigma^{-1} (x - \mu) + \log p(y | x) $ so $\log p(x|y)$ is concave
    \item not restricted to same type of distribution per time step (though in ISSSM it will be)
    \item Laplace approximation sensible for these types of models: single mode
    \item special case: exponential family distributions
\end{itemize}

The distribution of observations is never Gaussian - all statisticians may hope for is that the data-generating mechanism is close enough to a Gaussian distribution that inferences made carry over.
For epidemiological models, Gaussian distributions are appropriate if incidences are high, e.g. during large outbreaks in a whole country. 
When case numbers are small, the discrete nature of incidences is better captured by a distribution on $\mathbf N_{0}$, and standard distributions used are the Poisson and negative binomial distributions, see \todo{auto/cref?}.
Both the Poisson and negative binomial belong to the class of exponential family distributions. As such, their densities have a simple structure, allowing only for a linear interaction between the natural parameter and the densities argument. We refer to \cite{Brown1986Fundamentals} for a comprehensive treatment of exponential families and use their definitions throughout this chapter.

\begin{definition}[exponential family]
    Let $\mu$ be a $\sigma$-finite measure on $\R^{p}$ and denote by 
    $$\Theta = \left\{\theta \in \R^{p} : \int \exp \left( \theta^{T} x \right) \d\mu(y) < \infty\right\}$$
    the set of parameters $\theta$ such that the moment-generating function of $\mu$ is finite. 
    For every $\theta \in \Theta$ $$p_{\theta}(y) = Z(\theta)^{-1} \exp (\theta^{T} y)$$ defines a probability density with respect to the measure $\mu$, where $$Z(\theta) = \int \exp \left( \theta^{T} x \right) \d\mu(y)$$ is the normalizing constant. 
    We call both the densities $p_{\theta}$ and induced probability measures $$ \P_{\theta} (A) = \int_{A} p_{\theta}(y) \d \mu(y),$$ for measurable $A \subset \R^{p}$ \textbf{standard exponential families}.

    Conversely, let $\P_{\theta}, \theta \in \Theta$ be a given parametric family of probability measures on some space $\mathcal Y$ that is absolutely continuous with respect to a common dominating measure $\mu$. Suppose there exists a reparametrization $\eta : \Theta \to \R^{p}$, a statistic $T: \mathcal Y \to \R^{p}$ and functions $Z: \Theta\to \R$, $h:\mathcal Y \to \R$ exist, such that
    $$
        p_{\theta}(y) = \frac{\d P_{\theta}}{\d \mu} = Z(\theta) h(y) \exp \left(\eta(\theta)^{T}T(y)\right),
    $$
    then we call $\P_{\theta}, \theta \in \Theta$ and $p_{\theta}, \theta \in \Theta$ a \textbf{$p$-dimensional exponential family}.
\end{definition}

\todo{necessary?}
\begin{definition}[curved exponential family]

\end{definition}
    
Exponential families have the attractive property that they are log-concave in their parameters. As such the Fisher-information is always positive semidefinite, which will be crucial in defining surrogate Gaussian models in \todo{later section}.
\begin{lemma}[log-concavity of exponential family distributions]
    Let $p_{\theta}, \theta \in \Theta$ be a natural dimensional exponential family and $\Theta$ open in $\R^{p}$. In this case $\theta \to \log p_{\theta}(y)$ is concave for every $y \in \R^{p}$.
\end{lemma}

\begin{proof}
    As $\log p_{\theta}(y) = - \log Z(\theta) + \theta^{T} y$ it suffices to show that $\log Z(\theta)$ is convex. However, $\log Z(\theta)$ is the cumulant generating function of the base measure $\mu$ which is known to be convex \todo{more reasoning? differentiate under integral, check if dominated convergence applies, or look for a ref}.
\end{proof}

We now generalize \cref{def:glssm} to allow for non-gaussian observations by replacing the observation equation \cref{eq:glssm_observations} by more general exponential families.

\begin{definition}[\gls{lcssm}]
    A \textbf{\gls{lcssm}} is a state space model where states obey the transition equation 
    $$
    X_{t + 1} = A_{t}X_{t} + u_{t}\varepsilon_{t + 1}
    $$
    and the conditional distribution of $Y_{t}$ given $X_{t}$ comes from an exponential family with respect to a base measure $\mu_{t}$, i.e.
    $$
    p (y_{t}|x_{t}) = h_{t}(y_{t}) Z_{t}(x_{t}) \exp \left( \eta_{t}(x_{t})^{T} T_{t}(y_{t}) \right)
    $$
    for suitable functions $h_{t}, Z_{t}, \eta_{t}, T_{t}$. 

    \todo{LCSSM only logconcave observations would suffice, then EF is just an instance of this}
\end{definition}

\begin{remark}
    To simplify notation we will drop in our notation the dependence of $h$, $Z$, and $T$ on $t$ and assume that the base measure $\mu_t$ is the same for all relevant $t$.
\end{remark}

As in the previous chapter, after having observed $Y$, one is interested in the conditional distribution of states $X$, given $Y$. If the observations are not Gaussian, this is a difficult task as the distribution is not analytically tractable. Instead, one uses analytically tractable approximations such as the \gls{la}. Under appropriate assumptions on the observation distributions, the \gls{la} is given by a (possibly degenerate) \gls{glssm} \cite{Durbin1997Monte, Shephard1997Likelihood}, though the derivations and computations are more involved in the nonlinear case or if the observation densities are not log-concave.

\begin{theorem}[\gls{LA} of \gls{lcssm}]
    
\end{theorem}

\todo{this paragraph to modelling chapter}
The Poisson distribution arises from the law of small numbers: if there is a large population where every individual has, independently, a small probability of becoming infected in a small window of time then the total number of infections in that window of time is well approximated by the Poisson distribution.
Indeed, the law of small numbers remains valid for small dependencies \cite{Ross2011Fundamentalsa,Arratia1990Poisson}.
However, incidences observed from the SARS-CoV-2 epidemic tend to follow a negative binomial distribution \cite{Chan2021Count}. 



\section{Importance Sampling}
\label{sec:importance_sampling}

Suppose we have a function $h: \mathcal X \to \R$ whose integral $$\zeta = \int_{\mathcal X} h(x) \d x$$ we want to compute.
Furthermore suppose that we can write
$$
    \int_{\mathcal X} h(x) \d x = \int_{\mathcal X} f(x) \d \P(x)
$$
for a probability measure $\P$ and function $f: \mathcal X \to \R$.
Let $\G$ be another measure on $\mathcal X$ such that $f\P$ is absolutely continuous with respect to $\G$, $f\P \ll \G$ and let $v = \frac{\d f\P}{\d\G}$ be the corresponding Radon-Nikodym derivative. Then
$$
    \zeta = \int_{\mathcal X} h(x) \d x = \int_{\mathcal X} f(x) \d \P(x) = \int_{\mathcal X} v(x)\d\G(x)
$$
which suggests to estimate $\zeta$ by Monte-Carlo integration: $$\hat \zeta = \frac 1 N \sum_{i=1}^{N} v(X_i)$$ for $X_i \iid \G$, $i = 1, \dots, N$.

If one is not interested in a particular $h$ but rather in an approximation of $\P$ and $\P$ is absolutely continuous with respect to $\G$, then one may view $$\hat \P_N = \frac{1}{N} \sum_{i = 1}^{N} v(X_i) \delta_{X_i}$$ as a particle approximation of $\P$. In this setting \cite{Agapiou2017Importance} shows that the random measure $\hat \P_N$ converges to $\P$ at rate $\mathcal O\left(\frac 1 N\right)$\todo{check} in an appropriate metric.

To perform importance sampling one must be able to evaluate the weights $v$. In a bayesian setting this is usually infeasible: if $\P$ is a posterior then the integration constant of its density is intractable.
In this case one can usually evaluate the weights up to a constant, i.e. $w(x) \propto_x \frac{\d \P}{\d \G}(x)$ is available. The missing constant is then $\int w(x) \d \G$ which is itself amenable to importance sampling.

This leads to the self-normalized importance sampling weights $W_i = \frac{w(X_i)}{\sum_{i = 1}^N w(X_i)}$ and Monte Carlo estimates $\hat \zeta = \sum_{i = 1}^{N} W_i f(X_i)$ and particle approximation $\hat \P_N = \sum_{i = 1}^{N} W_i \delta_{X_i}$.

In both cases one can show that once second moments of $w$ with respect to $\G$ exist the Monte-Carlo estimates are consistent and asymptotically normal at the usual rates, see \cite[Chapter 8]{Chopin2020Introduction}.

Importance sampling is useful in situations where simulation from $\P$ is not feasible or when Monte Carlo integration with respect to $\P$ is unattractive due to high variance estimates.

\todo{decide where to put this}
As the likelihood of a general state space model is neither analytically nor numerically tractable one has to resort to Monte-Carlo techniques.
Recall that the likelihood is a high-dimensional integral of the form
\begin{align*}
    \lik(\theta) = p_\theta(y) = \int p_\theta(y,x) \d x = \int p_\theta(y|x) p_\theta(x) \d x = \E p_\theta(y|X).
\end{align*}
By the standard law of large numbers we can approximate $\lik(\theta)$ by
\begin{align*}
    \hat\lik(\theta) = \frac 1 N \sum_{i=1}^N p_\theta(y|X^i)
\end{align*}
for $N\in\N$ samples $X^i \iid p(x)$.
However, the variance of $\hat\lik(\theta)$ is likely to be very high if samples $X^i$ are drawn from the prior distribution $p(x)$ as they are not informed by the observations $y$.
As $p_\theta(x|y) \propto p_\theta(x,y)$ a more promising approach would be to use samples $X^i \sim p_\theta(x|y)$, but this distribution is usually not available.

While bayesian computational approaches such as MCMC\todo{cite} are able to generate (approximate) samples from this posterior distribution, importance sampling tries to find a distribution close to the target and re-weighs samples to ensure unbiased estimates of $\lik(\theta)$.

\begin{itemize}
    \item importance sampling as a variance reduction technique
    \item importance sampling as a technique to make intractable distributions tractable
    \item importance sampling vs. other methods:
          \begin{itemize}
              \item vs. ABC
              \item vs. MCMC
              \item vs. INLA (isn't this MCMC?)
          \end{itemize}
    \item measuring how good IS performs: ESS and other measures
    \item related results regarding performance of IS (Chatterje, Agapiou)
\end{itemize}

\subsection{Laplace approximation}

\begin{itemize}
    \item approximate at mode, problematic if posterior is not unimodal (but then gaussian approximation probably not worth it)
\end{itemize}
\subsection{Cross entropy method}
\begin{itemize}
    \item
\end{itemize}
\subsection{Efficient importance sampling}


\section{Gaussian importance sampling for state space models}
\label{sec:gaussian_importance_sampling_for_state_space_models}

Most models in this thesis can be viewed as an inverse problem of the form

\begin{align*}
    \mathbf{R}^{n\cdot m} \ni X & \sim \mathcal N(\mu, \Sigma) \\
    Y|X                         & \sim Y|BX \sim p(y|s)
\end{align*}
and the state space formulation allows for efficient computation of, e.g., $p(y|x)$.
To perform importance sampling for the smoothing distribution $p(x|y)$ we want to have close tractable approximations, that also depend on few parameters, ideally only $\mathcal O(n)$ many.

\todo{paragraph about laplace approximation for the posterior}

In total we want to perform importance sampling with proposal distributions $g(x|z)$ given by Gaussian linear models of the form

\begin{align*}
    X    & \sim \mathcal N(\mu, \Sigma) \\
    Z    & = BX + \eta                  \\
    \eta & \sim \mathcal N(0, \Omega).
\end{align*}
The dependency structure of the state space model implies that $\Omega$ should be a blockdiagonal matrix with at most $n \cdot m^{2}$ many non-zero entries.
If, additionally, the observations $y_{t}$ are conditionally independent given $x_{t}$, i.e. if $p(y_{t}|s_{t}) = \prod_{i = 1}^{p} p(y_{t}^{i}|s_{t}^i)$, then $\Omega$ is a diagonal matrix with only $\mathcal O(n \cdot m)$ many non-zero entries.

The proposal distribution $g(x|z)$ is then parameterized by the synthetic observations $z$ and the entries of $\Omega$ and we denote this set of parameters by $\psi = (z, \Omega)$.
The following results on this distribution will be useful when analysing Gaussian importance sampling.

\todo{move to other subsection}
The Laplace approximation chooses $\psi_{\text{LA}}$ such that the mode of $g(x|z)$ and the curvature at the mode match that of the true posterior, while the CE-method and EIS choose $\psi_{\text{CE}}$ and $\psi_{\text{EIS}}$ the solutions to associated optimization problems.

This means that we can treat all three methods in the same framework, facilitating comparison between the resulting three importance sampling proposals.


\subsection{Gaussian smoothing proposals}
\label{subsec:gaussian_smoothing_proposals}
In this section, we analyze the properties of Gaussian proposals for importance sampling in \glspl{ssm} that exploit the available Markov property of states. As mentioned in the introduction to this section, these proposals are conditional distributions $\P^{X|Z=z}$ where $ \mathbf{R}^{m} \ni X \sim \mathcal N \left( \mu, \Sigma \right)$ and $Z = BX + \eta \in \mathbf{R}^{p}$ where $\eta\sim\mathcal N(0, \Omega)$ is independent of $X$. Standard results from linear regression theory imply that the conditional distribution in question is again a Gaussian distribution, $X|Z=z \sim \mathcal N(\bar \mu, \bar \Sigma)$ with mean
\begin{align}
    \bar \mu    & = \mu + \Sigma B^{T} \left( B \Sigma B^{T} + \Omega \right)^{-1} (z - B \mu), \label{eq:posterior_mean_1} \\
                & = \bar \Sigma\left(\Sigma^{-1}\mu + B^{T}\Omega^{-1}z \right)\label{eq:posterior_mean_2}                  \\
    \intertext{and covariance matrix}
    \bar \Sigma & = \Sigma - \Sigma B^T \left( B\Sigma B^{T} + \Omega \right) ^{-1} B \Sigma \label{eq:posterior_cov_1}     \\
                & = \left(\Sigma^{-1} + B^{T}\Omega^{-1}B\right)^{-1} \label{eq:posterior_cov_2}.
\end{align}
Note that \Cref{eq:posterior_mean_1,eq:posterior_cov_1} are more general, requiring only $B \Sigma B + \Omega$ be invertible, while the others require both $\Sigma$ and $\Omega$ to be invertible, see \cite[Lemma 7.1]{Chopin2020Introduction} for further discussion.

\begin{proposition}[Exponential family of smoothing distribution]
    \label{prop:exponential_family_posterior}
    Suppose $\Omega$ is invertible.
    In this case the family of conditional distributions $X|Z=z$ parameterized by $z$ and $\Omega$ form an exponential family

    \begin{align*}
        p(x|z) & = h(x) \exp \left( \left\langle \eta, T(x) \right\rangle - A(\eta)\right)
    \end{align*}
    where the parameters are 
    \begin{align*}
      \eta &= \left( \eta_{1}, \eta_{2} \right) = \left( \bar\Sigma^{-1} \bar\mu, -\frac{1}{2} \Omega^{-1}\right) \\
      \intertext{and }
      h(x) &= \frac{1}{\sqrt{(2\pi)^{m} \det \Sigma}}\exp \left( - \frac{1}{2} x^{T} \Sigma^{-1}x \right) \\
      A(\eta) &= \frac{1}{2} \left( \log \det \left( I - \Sigma \,\diag \left( 2 \eta_{2} \right)  \right)  + \bar\mu^{T}\bar\Sigma^{-1}\bar\mu \right) \\
      &= \frac{1}{2} \log \det \left(I - \Sigma \, \diag\left(2\eta_{2}\right)\right) + \frac{1}{2} \eta_{1}^{T} \left( \Sigma^{-1} - \diag (2\eta_{2}) \right)  \eta_{1}\\
      T(x) &= \left( x, x x ^T \right).
    \end{align*}

    Note that $\eta_{1} = \Sigma ^{-1} \mu + B^{T}\Omega^{-1}z \in \Sigma^{-1}\mu + \operatorname{im} B^{T}$ making the exponential family curved if $\operatorname{rank} B < m$.
    \todo{fix $\Omega$ and $\diag{\omega}$ here}
\end{proposition}
\todo{probably cite something about curved exponential families, e.g. Brown1986Fundamentals}

\subsection{Analysis of optimal parameters}
\label{subsec:analysis_of_optimal_parameters}


\begin{theorem}[Optimal EIS proposal]
    \label{thm:optimal-eis}
    Let $p(x)$ be some density and consider importance sampling by exponential family proposals with densities $$q_\psi(x) = h(x) \exp\left( \langle \psi,S(x)\rangle - A(\psi)\right)$$ with natural parameter $\psi \in \mathbf{R}^{k}$, base measure $h$, sufficient statistic $S$ and log-partition function $A$. The parameter $\hat \psi$ that minimizes the variance of log importance sampling weights $\log w_{\psi}(x) = \log p(x) - \log q_{\psi}(x)$ is given by
    \begin{align*}
        \hat \psi & =  \argmin _{\psi} \var \left( \log w_{\psi}(X) \right)        \\
                  & = \cov(S(X))^{-1}\cov\left(S(X), \log \frac{p(X)}{h(X)}\right)
    \end{align*}
    where $X \sim p$.
\end{theorem}
\begin{proof}
    \todo{formultae this, consider exact assumptions}
\end{proof}

\begin{remark}[Optimal Gaussian proposal]
    As the family of Gaussian distributions $\mathcal N \left( \mu, \Sigma \right)$ form an exponential family with natural parameter $\psi = \left( \Sigma ^{-1} \mu, -\frac{1}{2}\Sigma^{-1} \right)$ and sufficient statistic $S(x) = \left( x, x x^{T} \right)$, \Cref{thm:optimal-eis} implies that the optimal EIS Gaussian proposal involves up to fourth order moments of $p$.

    As a consequence we expect EIS to produce proposals that are more robust to skewness and heavier than Gaussian tails than the Laplace approximation \todo{which is validated by simulations in section ...}.
\end{remark}

\subsection{Analysis of convergence (?)}
\label{subsec:analysis-of-convergence}

Additionally, each iteration of the CE and EIS method may be seen as performing M-estimation and as such the one step estimates $\psi_{CE}$ and $\psi_{EIS}$ are, in the limit as the number of samples $M$ goes to $\infty$, asymptotically normally distributed.

Analyzing the multi-step behavior of these iterative estimates is more complex, as we want to keep a fixed seed, i.e. common random numbers, to ensure numerical convergence. Thus the distribution of the second iterate conditional on the first iterate depends only \todo{check} the conditional distribution of the common random numbers given the first iterate, which is intractable.

\begin{theorem}[Consistency of importance sampling estimates]
    \todo{apply van der vaart}
\end{theorem}

\begin{theorem}[Asymptotic normality of importance sampling estimates]
    \todo{calculate asymptotic covariances}
\end{theorem}
\begin{proof}
    \todo{all iterative procedures are M-estimators, so a single step is (in the limit of samples $N\to\infty$), under some regularity conditions, asymptotically normal, compare asymptotic variances}
\end{proof}

\todo{interpret this in a sensible way, probably EIS more numerically stable}

\section{Accouting for multimodality and heavy tails}
\label{sec:accouting_for_multimodality_and_heavy_tails}
Performing importance sampling with the Gaussian models discussed so far will work well only if the smoothing distribution  $p(x|y)$ is well approximated by a Gaussian distribution. However, a Gaussian distribution is a very specific kind of distribution, in particular, it is an unimodal distribution
%that is constant on elliptical contours 
and has light tails \todo{check for correct wording}.

If the smoothing distribution violates any of these assumptions, importance sampling with the models presented so far is likely to fail, i.e. requiring large sample sizes for both finding the optimal importance sampling parameter $\hat \psi$ as well as the final importance sampling evaluation.

There are however techniques to keep most of the computational efficiency discussed in the above sections to address both multimodality as well as heavy tails.

We start with heavier than gaussian tails: the textbook example of a heavy tailed distribution is the multivariate $t$-distribution with density
$$
    \dots .
$$
for degrees of freedom  $\nu > 1$ \todo{?}, location $\mu$ and scale matrix $\Sigma$. When $\nu > 2$ then this distribution has mean $\mu$ and if $\nu > 3$ it has covariance matrix $?$ \todo{check}.

The main properties necessary to facilitate Gaussian importance sampling strategies above are that the distribution $p(x|y)$ is analytically tractable and simulation from it is possible. These properties still hold for the multivariate $t$-distribution and, in fact, for the even larger class of elliptical distributions:

\begin{theorem}[Conditional distribution of elliptical distributions]
    \label{thm:elliptical-conditional}
    \todo{cite the correct book}
\end{theorem}

As one can readily see from \Cref{thm:elliptical-conditional} the parameters of the smoothing distribution $p(x|y)$ if $p(x,y)$ follows an elliptical distribution is again elliptical and its parameters only depend on quantities that are computed by the Kalman smoother. \todo{elaborate}

\todo{present some models with heavy tails}


\section{Maximum likelihood estimation}
\label{sec:maximum_likelihood_estimation}



%\section{State space models in high dimensions}
%\label{sec:state_space_models_in_high_dimensions}
