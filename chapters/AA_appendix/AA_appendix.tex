\chapter{Reproducibility and code}
\label{cha:reproducibility_and_code}
All code used in to create figures and tables in this thesis is written in Python and R \todo{cite} and available as open source software. Python is used for simulations, while R is used to create figures and tables of these results. 

The code is split into two software packages:
\begin{itemize}
    \item Importance Sampling for State Space Models (\texttt{isssm})\footnote{\url{https://stefanheyder.github.io/isssm}} is a Python package developed by the author. It implements frequentist inference for \acrshortpl{ssm} using the general methods described in this thesis, in particular the \acrshort{cem} and \acrshort{eis} for \acrshortpl{pgssm}. 
    \item The \acrshortpl{ssm} for Epidemiology \texttt{ssm4epi} package contains Python and R code particular to this thesis, i.e. the code needed to reproduce all results and figures in this thesis. 
\end{itemize}

The \texttt{ssm4epi} package is available as Jupyter Notebooks organized by chapters of this thesis. To reproduce the results of this thesis, follow the instructions in the associated documentation \todo{ref to doc}. Simulations use a fixed seed that is set at the beginning of each notebook to ensure reproducibility. 

The data produced by these Jupyter notebooks are available on zenodo \todo{put them there}, and can be reproduced by running the notebooks. Figures and tables in this thesis that depend on simulation results can be reproduced similarly, using Jupyter notebooks with an R kernel. Dependent R packages can be found in the \texttt{setup.R} file in this thesis' GitHub repository.

\chapter{Additional calculations}



\section*{\Cref{eq:variance-wxx}}
To show \Cref{eq:variance-wxx} we calculate the second moment of $w(X)X$,
\begin{align*}
    \mathbb E (w(X)X)^{2} &= \int w(x)^{2} x^{2} g(x) \d x \\
    &= \int \sigma^{2} \exp \left( - x^{2} \left( 1 - \frac{1}{\sigma^{2}} \right)\right)\, x^{2}\, \frac{1}{\sqrt{2\pi\sigma^{2}}} \exp \left( -\frac{x^{2}}{2\sigma^{2}} \right)  \d x \\
    &= \int \sigma x^{2} \frac{1}{\sqrt{2\pi}}\exp \left( - \frac{x^{2}}{2} \left( 2 - \frac{2}{\sigma^{2}} + \frac{1}{\sigma^{2}} \right) \right)\d x \\
    &= \int \sigma x^{2} \frac{1}{\sqrt{2\pi}}\exp \left( - \frac{x^{2}}{2} \frac{2\sigma^{2} - 1}{\sigma^{2}} \right)\d x \\
    &= \tau \sigma \int x^{2} \frac{1}{\sqrt{2\pi\tau^{2}}} \exp \left( - \frac{x^{2}}{2\tau^{2}}\right)\d x \\
    &= \tau^{3} \sigma = \frac{\sigma^{4}}{(2\sigma^{2} - 1)^{\frac{3}{2}}}
\end{align*}
where $\tau^{2} = \frac{\sigma^{2}}{2\sigma^{2} - 1}$.

\section*{\Cref{ex:univ-gaussian-mu-fixed}}
We have $I(\psi_{\ce})^{-1} = \frac{ \left(2\mu^{2} + 2\tau^{2}\right)^{2}}{2} $ by standard properties of the single parameter Gaussian exponential family, so $B_{\ce} = \frac{2}{\left(2\mu^{2} + 2\tau^{2}\right)^{2}}$. 

Additionally,
\begin{align*}
M_{\ce} = \cov_{\P} (T) & = \P \left( \left((\id - \mu)^{2} - \tau^{2} - \mu^{2} \right)^{2}\right)\\
    &= \P \left( \left( \id^{2} - \tau^{2} + 2\mu\id \right)^{2} \right) \\
    &= \P \left(\id^{4} - 2\id^{2}\tau^{2} + 4 \mu \id^{3} + \tau^{4} + 4\mu^{2}\id^{2}\right)\\
    &= \nu + 4 \mu^{2}\tau^{2} - \tau^{4},
\end{align*}
as $\P \id^{3} = 0$ and $\P \id^{2} = \tau^{2}$. In total 
$$
    V_{\ce} = B_{\ce} M_{\ce} B_{\ce} = \frac{4 \left(\nu + 4 \mu^{2}\tau^{2} - \tau^{4}\right)}{\left(2\mu^{2} + 2\tau^{2}\right)^{4}} = \frac{\nu + 4 \mu^{2}\tau^{2} - \tau^{4}}{4 \left( \mu^{2} + \tau^{2}\right)^{4}}.
$$

For \acrshort{eis}, we have $\log p(x) = -\frac{1}{2} \frac{x^{2}}{\tau^{2}} - \frac{1}{2\sqrt{\pi\tau^{2}}}$, so the log-weights are, up to an additive constant, given by
$$
    - \frac{1}{2} \frac{\id^{2}}{\tau^{2}} - T \psi_{\eis}.
$$

$M_{\eis}$ is then given by 
$$
    \P \left( (\id - \mu)^{4} \left( \log w - \P \log w \right)^{2}  \right)
$$
which is the expectation of a sixth order polynomial with respect to the standard normal distribution $\P$, so its value is analytically tractable and turns out to be\footnote{See \texttt{03\_08\_comparison.ipynb} for calculations.}
$$ M_{\eis} = \frac{\mu^{2} \left(2 \mu^{6} + 45 \mu^{4} \tau^{2} + 15 \tau^{6}\right)}{\left(2 \mu^{2} + \tau^{2}\right)^{2}}.
$$


