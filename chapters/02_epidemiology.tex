\glsresetall
\chapter{Epidemiological considerations}
\label{chap:epidemiological_considerations}
\begin{tcolorbox}[title={Contributions of this chapter}]
    \begin{itemize}
        \item ...
    \end{itemize}
\end{tcolorbox}
\newpage

%this chapter highlights challenges that epi modelling brings and what desirable outcomes would be from an applied perspective
The spread of infectious diseases, such as \acrshort{c19}, is a complex phenomenon. For \acrshort{c19} this complexity arises from the interplay of many factors: First of all, there is considerable heterogeneity in the way the disease progresses once an individual is infected \cite{Salzberger2021Epidemiology} \todo{better ref?}. Some infectees may show few to no symptoms but are still highly infectious \cite{Byambasuren2020Estimating}, and disease progression is tightly linked to age and preexisting comorbidities \cite{Biswas2020Association}. Additionally, different variants of \acrshort{scov2} differ in key characteristics such as the reproduction number \cite{Du2022Reproduction} and mortality \cite{Hughes2023Effect}. 

Second, the spread is highly dependent on the contact behavior in the population, as the infector has to be in close physical proximity to the infectee to infect them. 
% immunity levels in population
% contact behavior
% self-regulation
% political intervention


%mathematical epidemiology concerns itself with modelling epidemiolocial systems, from small (local outbreaks) to large (epi/pandemics)
%conclusions from analysis only as good as the model and the data are
%dependening on goal and circumstances different methods are applicable
%by its nature, data are observational so causal claims difficult
%in this thesis I focus on models for larger-scale epidemics, techniques would be flexible enough to deal with smaller scale as well, as long as latent states are gaussian

