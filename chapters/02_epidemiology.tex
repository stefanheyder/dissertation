\glsresetall
\chapter{Epidemiological considerations}
\label{chap:epidemiological_considerations}
\begin{tcolorbox}[title={Contributions of this chapter}]
    \begin{itemize}
        \item ...
    \end{itemize}
\end{tcolorbox}
\newpage

The spread of infectious diseases, such as \acrshort{c19}, is a complex phenomenon. For \acrshort{c19} this complexity arises from the interplay of many factors. Studying these influences allows us to define the aims and challenges of epidemiological modeling in the context of this thesis. It also will guide us towards desirable and possible outcomes of our efforts from an applied perspective.

First of all, there is considerable heterogeneity in the way the disease progresses once an individual is infected \cite{Salzberger2021Epidemiology} \todo{better ref?}. Some infectees may show few to no symptoms but are still highly infectious \cite{Byambasuren2020Estimating}, and disease progression is tightly linked to age and preexisting comorbidities \cite{Biswas2020Association}. Additionally, different variants of \acrshort{scov2} differ in key epidemiological characteristics such as the reproduction number \cite{Du2022Reproduction} and mortality \cite{Hughes2023Effect}. 

Second, the spread is highly dependent on the contact behavior in the population, as the infector has to be in close physical proximity to the infectee to infect them. These contact patterns are an essential component of any mathematical model for infectious diseases, as they define how the epidemic evolves. While there are some empirical studies \cite{Tomori2021Individual,Mossong2008Social}, capturing the contact behavior at certain points in time, in the context of an ongoing epidemic these patterns are subject to change, not only in intensity but also in shape \cite{Tomori2021Individual}. As contact restrictions are put into place or lifted, mask-wearing is enforced and home office and schooling became commonplace, so did the number of contacts change and occur at different places.

Finally, as the virus spread in the population and vaccinations became available, the population became partially immune against the disease, if not against infection. This immunity affects the spread as well: if an infector has contact with a partially immune individual, the probability of transmission is smaller. Additionally, partial immunity may lead infectors to develop fewer or no symptoms, and so they may not be aware of being infectious. \todo{cite?}

As statisticians, we are faced with a difficult problem: Which of these factors should we include in our model and how? The answer certainly depends on the epidemiological question under consideration and the availability and quality of data. 

%mathematical epidemiology concerns itself with modelling epidemiolocial systems, from small (local outbreaks) to large (epi/pandemics)
%conclusions from analysis only as good as the model and the data are
%dependening on goal and circumstances different methods are applicable
%by its nature, data are observational so causal claims difficult
%in this thesis I focus on models for larger-scale epidemics, techniques would be flexible enough to deal with smaller scale as well, as long as latent states are gaussian

\section{Objectives of epidemiological modelling}

Before considering the mathematical modeling of epidemics, let us make clear what the goals of our investigation are. In this thesis, we are interested in providing models that are informed by real-world data, allow us to learn about the past, current or future state of the epidemic and whose results are, ideally, easy to communicate to non-experts, e.g. political stakeholders.
These time scales can be translated into the following three tasks for epidemiological modeling.

\paragraph{Retrospective Analysis}
Here we are interested in an ex-post analysis of a period of interest in the past. The goal here is either to infer intrinsic epidemiological quantities, such as the time-varying reproduction number $R_{t}$ \cite{Abbott2020Estimating} or to evaluate the performance of \glspl{npi} taken \cite{Flaxman2020Estimating,Brauner2021Inferring,Khazaei2023Using}. The results of this analysis may inform future decisions on which countermeasures to implement, and as such we want a causal link between the \acrshortpl{npi} prescribed and the reduction in reported cases. Naturally, this is a difficult objective to accomplish due to several aspects. The data at our disposal is observational and there are several quality issues, see \Cref{sec:data}. Additionally, the interplay between \acrshortpl{npi} and change in the behavior of the population is intricate, where voluntary behavioral change may precede the enforced social distancing \cite{Gupta2020Mandated}. 
For some examples focusing on the efficacy of \acrshortpl{npi}, we refer the reader to the excellent articles \cite{Flaxman2020Estimating,Brauner2021Inferring,Khazaei2023Using}, especially the discussion and limitation sections therein. 
In these types of analyses, we can assume that all data related to that period is as complete as it will be. 
Methods used to perform these analyses range from estimating parameters for each day individually, e.g. using the EpiEstim \cite{Cori2021EpiEstim} method \cite{Abbott2020Estimating}, to constructing complex Bayesian mechanistic \cite{Flaxman2020Estimating} and hierarchical models \cite{Brauner2021Inferring,Khazaei2023Using}. 
% methods
%% Flaxman: Bayesian mechanistic model w/ partial pooling of interventions 
%% Brauner: Flaxman + incidences, more data, priors on epi parameters, NPI difference across countries
%% Khazei: Bayesian hierarchical model
%% Abbott: EpiEstim, each day separately

\paragraph{Monitoring}
For monitoring, we are interested in real-time inference about the current state of the epidemic. This includes the recent past and near future and may include now- and forecasts of cases, hospitalizations or deaths. Here data is not yet final, and inference is complicated by slow reporting and data revisions, see \Cref{sec:data}. The results of monitoring can be used to inform current policy, i.e. whether current \acrshortpl{npi} should be lifted or new ones enforced. Most online dashboards that emerged at the beginning of the pandemic fall into this category. The result of monitoring may either be an estimate of an epidemiological indicator, but may also consist of short-term forecasts. Examples of the former include the daily reproduction number estimates of the \acrshort{rki} \cite{AnDerHeiden2020Schatzung}, the Helmholtz Centre for Infection Research's dashboard \cite{Khailaie2021Development} or the dashboard of the authors team \cite{Hotz2020Monitoring}.
While some of these dashboards also provide forecasts of cases, a more concerted effort of forecasts is provided by the U.S. ForecastHub \cite{Ray2020Ensemble}, its German/Polish \cite{Bracher2021Preregistered,Bracher2022National} and EU/EFTA \cite{Sherratt2022Predictive} equivalents. These collaborative platforms gathered real-time forecasts of \acrshort{c19} cases and deaths in the upcoming four weeks, based on an ensemble that aggregates predictions from several models provided by expert modelers. In a real-time setting, these forecasts can be evaluated which may inform practitioners as to which model to prefer. 
For forecasting, methods range from classical time series analysis methods \cite{Arroyo-Marioli2021Tracking} to compartmental models \cite{Khailaie2021Development} and computationally intensive agent-based models \cite{Adamik2020Mitigation}.

\paragraph{Scenario Modeling}
Scenario modeling concerns itself with the impact that changes of current circumstances, e.g. variants, seasonality, policies, vaccination or \acrshortpl{npi}, have on public health outcomes. Contrary to monitoring, the goal is to quantify the influence over longer periods with scenarios reaching multiple months into the future. The parameters of scenarios are assumed to be uncertain as well, making the task at hand challenging. These forecasts are difficult to evaluate, as the scenario specifications rely on assumptions that are hard to verify in practice. Nevertheless, these scenarios help policymakers make informed decisions \cite{Borchering2023Public}.

In this thesis, we will forego scenario modelling
\section{Available data and its quality}
\label{sec:data}

\begin{itemize}
    \item surprising amount of data available, but quality questionable, 
    \item in Germany have data on reported cases and deaths by gender, age group, county, with reporting date of case and for some cases even date of symptom onset
    \item reporting of cases is regulated by Infektionsschutzgesetz
    \item parallel dataset for reports of hospitalisations 
    \item have description section from Nowcasting draft here
    \item descriptive statistics of German COVID-19 data set
    \item even larger datasets that compile this for europe + EFTA (?) by ECDC or by world (JHU)
    \item quality of reported case data is potentially too low 
    \begin{itemize}
        \item reporting delays
        \item weeakday effects
        \item testing regime changing (2G/3G)
        \item ...
    \end{itemize}
    \item data on commuting
\end{itemize}
\section{Measures of epidemic spread}
\label{sec:measures_of_epidemic_spread}
This section consists of the ideas published in \cite{Heyder2023Measures}, but has been rewritten to fit better into this thesis.

\begin{itemize}
    \item not only epidemic spread but also speed of proliferation is of interest, enables forecasts
    \item measuring speed difficult: data problems ... (look at AK book article)
    \item 
\end{itemize}

\subsection{Growth Factor}
\label{subsec:growth_factor}
\subsection{Reproduction number}
\label{subsec:reproduction_number}

\subsection{Other indicators}
\label{subsec:other_indicat}

\subsection{Usefulness of indicators}
\label{subsec:usefulness_of_indicators}


\section{Desiderata for epidemiological models}
\label{sec:dessiderata}

As the dynamics of the epidemic are constantly changing, our models should

\todo{mention problems with too simple R estimate in local outbreak setting (over / under estimation)}
% slide 3 in file:///Users/stefan/workspace/work/talks/2022-07-25%20Siegmundsburg2022%20RegionalReproductionNumbers/Heyder2022SiegmundsburgRegional.pdf
% incorporate points from spatio temprial R slide therein as well


\begin{itemize}
    \item we want models to be able to include as much data as possible, while still being numerically tractable 
\end{itemize}

\todo{this paragraph to modelling chapter}
The Poisson distribution arises from the law of small numbers: if there is a large population where every individual has, independently, a small probability of becoming infected in a small window of time then the total number of infections in that window of time is well approximated by the Poisson distribution.
Indeed, the law of small numbers remains valid for small dependencies \cite{Ross2011Fundamentalsa,Arratia1990Poisson}.
However, incidences observed from the SARS-CoV-2 epidemic tend to follow a negative binomial distribution \cite{Chan2021Count}. 

\paragraph{Regional dependencies and effects}
% use data to inform this section
\begin{itemize}
    \item German case data are reported on Landkreis level, performing analysis of each individual is not sensible 
    \item inhabitants travel between regions, and measures were taken on on regional level as well
    \item effects are not really spatial: euclidean distance is not so much of an issue but how closely connected regions are (give some examples)
    \item also want to account for other regional effects such as different socio-economic settings ... 
\end{itemize}

\paragraph{Temporal correlation}

\paragraph{Interpretability}


