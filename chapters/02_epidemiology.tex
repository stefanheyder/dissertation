\glsresetall
\chapter{Epidemiological considerations}
\label{chap:epidemiological_considerations}

\begin{itemize}
    \item COVID-19 induced unprecedented interest in epidemiological modelling from all disciplines, but also mathematics
    \item this chapter highlights challenges that epi modelling brings and what desirable outcomes would be from an applied perspective
    \item mathematical epidemiology concerns itself with modelling epidemiolocial systems, from small (local outbreaks) to large (epi/pandemics)
    \item conclusions from analysis only as good as the model and the data are
    \item dependening on goal and circumstances different methods are applicable
    \item by its nature, data are observational so causal claims difficult
    \item in this thesis I focus on models for larger-scale epidemics, techniques would be flexible enough to deal with smaller scale as well, as long as latent states are gaussian
\end{itemize}
\section{Objectives of epidemiological modelling}

\paragraph{Monitoring}
\begin{itemize}
    \item monitoring is real-time scenario, interested in current developments, i.e. recent past and near future. complicated by potentially slow reporting, data revisions
    \item informs decision makers on whether measures should be taken
    \item ForecastHub(s) provide platform that creates ensemble forecast to obtain better predictions \cite{Bracher2022National,Bracher2021Preregistered,Ray2020Ensemble,Sherratt2022Predictive}
\end{itemize}
\paragraph{Retrospective Analysis}
\begin{itemize}
    \item evaluation of measures taken, want interpretation as causal as possible
    \item informs decision makes on which measures were effective and how much
    \item difficult due to usual reasons: poor data quality, observational data, causual structure difficult, early/late adoption makes timing of measurements difficult
    \item cite some papers that did this \cite{Flaxman2020Estimating,Brauner2021Inferring,Khazaei2023Using}
\end{itemize}
\paragraph{Scenario Modelling}

\begin{itemize}
    \item concerns itself with modelling the impact that variants, seasonality etc. have in specific scenarios 
    \item find out whether there is already paper of ECDC to cite
\end{itemize}

\section{Available data and its quality}
\label{sec:data}

\begin{itemize}
    \item surprising amount of data available, but quality questionable, 
    \item in Germany have data on reported cases and deaths by gender, age group, county, with reporting date of case and for some cases even date of symptom onset
    \item reporting of cases is regulated by Infektionsschutzgesetz
    \item parallel dataset for reports of hospitalisations 
    \item have description section from Nowcasting draft here
    \item descriptive statistics of German COVID-19 data set
    \item even larger datasets that compile this for europe + EFTA (?) by ECDC or by world (JHU)
    \item quality of reported case data is potentially too low 
    \begin{itemize}
        \item reporting delays
        \item weeakday effects
        \item testing regime changing (2G/3G)
        \item ...
    \end{itemize}
    \item data on commuting
\end{itemize}

\section{Measures of epidemic spread}
\label{sec:measures_of_epidemic_spread}
This section consists of the ideas published in \cite{Heyder2023Measures}, but has been rewritten to fit better into this thesis.

\begin{itemize}
    \item not only epidemic spread but also speed of proliferation is of interest, enables forecasts
    \item measuring speed difficult: data problems ... (look at AK book article)
    \item 
\end{itemize}

\subsection{Growth Factor}
\label{subsec:growth_factor}
\subsection{Reproduction number}
\label{subsec:reproduction_number}

\subsection{Other indicators}
\label{subsec:other_indicat}

\subsection{Usefulness of indicators}
\label{subsec:usefulness_of_indicators}


\section{Dessiderata for epidemiological models}
\label{sec:dessiderata}

\begin{itemize}
    \item we want models to be able to include as much data as possible, while still being numerically tractable 
\end{itemize}

\paragraph{Regional dependencies and effects}
\begin{itemize}
    \item German case data are reported on Landkreis level, performing analysis of each individual is not sensible 
    \item inhabitants travel between regions, and measures were taken on on regional level as well
    \item effects are not really spatial: euclidean distance is not so much of an issue but how closely connected regions are (give some examples)
    \item also want to account for other regional effects such as different socio-economic settings ... 
\end{itemize}

\paragraph{Temporal correlation}

\paragraph{Interpretability}



