\section{Objectives of epidemiological modelling}
\label{sec:objectives}

Before considering the mathematical modeling of epidemics, it's important to clarify the goals of our investigation are. In this thesis, we are aim at providing models informed by real-world data, that allow us to learn about the past, current or future state of the epidemic and whose results are, ideally, easy to communicate to non-experts, such as political stakeholders.
These time scales can be translated into the following three tasks for epidemiological modeling:

\paragraph{Retrospective Analysis}
Here we are interested in an ex-post analysis of a period of interest in the past. The goal is to infer intrinsic epidemiological quantities, such as the time-varying reproduction number $R_{t}$, e.g. as in  \citep{Abbott2020Estimating}, or to evaluate the performance of \glspl{npi} taken, see e.g. \citep{Flaxman2020Estimating,Brauner2021Inferring,Khazaei2023Using}. This analysis can inform future decisions on which countermeasures to implement, highlighting causal links between \acrshortpl{npi} prescribed and reductions in reported cases. This objective is challenging, due to the observational nature of the data quality issues, see \Cref{sec:data}. The interplay between \acrshortpl{npi} and voluntary behavior change the population adds complexity, as voluntary behavioral changes may precede enforced social distancing \citep{Gupta2020Mandated}. 

For examples focusing on the efficacy of \acrshortpl{npi}, we refer the reader to \citep{Flaxman2020Estimating,Brauner2021Inferring,Khazaei2023Using}, especially the discussion and limitation sections therein. 
In such analyses, we can assume that all data related to that period is as complete as possible.
Methods range from estimating parameters for each day individually, e.g. using the EpiEstim \citep{Cori2021EpiEstim} method \citep{Abbott2020Estimating}, to constructing complex Bayesian mechanistic \citep{Flaxman2020Estimating} and hierarchical models \citep{Brauner2021Inferring,Khazaei2023Using}. 

\paragraph{Monitoring}
For monitoring, we are interested in real-time inference about the current state of the epidemic. This includes the recent past and near future and may include nowcasts and forecasts of cases, hospitalizations or deaths. In this setting, data is not yet final, and inference is complicated by slow reporting and data revisions, see \Cref{sec:data}. The results of monitoring can be used to inform current policy, i.e. whether current \acrshortpl{npi} should be lifted or new ones enforced. Most online dashboards that emerged at the beginning of the pandemic fall into this category. The result of monitoring may be an estimate of an epidemiological indicator, but may also consist of short-term forecasts. Examples of the former include the daily reproduction number estimates of the \acrshort{rki} \citep{AnDerHeiden2020Schatzung}, the Helmholtz Centre for Infection Research's dashboard \citep{Khailaie2021Development} or the dashboard of this thesis' authors team \citep{Hotz2020Monitoring}.

While some of these dashboards also provide forecasts of cases, a more concerted effort of forecasts is provided by the U.S. ForecastHub \citep{Ray2020Ensemble}, its German/Polish \citep{Bracher2021Preregistered,Bracher2022National} and EU/EFTA \citep{Sherratt2022Predictive} equivalents. These collaborative platforms gathered real-time forecasts of \acrshort{c19} cases and deaths in the upcoming four weeks, based on an ensemble that aggregates predictions from several models provided by expert modelers. In a real-time setting, these forecasts can be evaluated without hindsight bias, which informs practitioners as to which model to prefer. 
For forecasting, methods can range from classical time series analysis methods \citep{Arroyo-Marioli2021Tracking} to compartmental models \citep{Khailaie2021Development} and computationally intensive agent-based models \citep{Adamik2020Mitigation}.

\paragraph{Scenario Modeling}
Scenario modeling concerns itself with the impact that changes of current circumstances, e.g. variants, seasonality, policies, vaccination or \acrshortpl{npi}, have on public health outcomes. Unlike monitoring, the goal is to quantify the influence over longer periods with scenarios reaching multiple months into the future. The parameters of scenarios are assumed to be uncertain, making the task at hand challenging. These forecasts are difficult to evaluate, as the scenario specifications rely on assumptions that are hard to verify in practice. Nevertheless, these scenarios help policymakers make informed decisions \citep{Borchering2023Public}.\\[20pt]
%
In the context of this thesis, we are primarily interested in performing retrospective analyses and providing tools for monitoring as well as short-term forecasting. While scenario modeling has its own merits, evaluating the performance of models is much harder, as there is no ground truth for comparison. Additionally, the methods developed in this thesis rely on having recurring observations on a daily or weekly time scale, usually in the form of reported cases, deaths, or hospitalizations, which for scenario modeling are not available. If such observations are not available, i.e. because we are forecasting months ahead, the uncertainty produced by our models will be much too large to be sensible. To produce short-term forecast, we need to specify how fast the epidemic is spreading.