
\section{Modeling epidemiological desiderata with state space models}
\label{sec:modelling_epidemiological_dessiderata_with_state_space_models}

Now that we have specified the class of models that we are going to use in this thesis, let us demonstrate that we can model many of the phenomena identified in \Cref{sec:dessiderata} in \acrshortpl{ssm}. A major strength of \acrshortpl{ssm} is that modeling non-stationary time series data is straightforward, as $p(x_{t}|x_{t - 1})$ can be an arbitrary density. Indeed, as the states $X$ are unobserved, they can be used to model relevant, but unobserved or unavailable variables, that affect the observed data $Y$. Thus, $Y$ will be th

% major strength of SSMs:
% blurb about X hidden, allows to model unobserved variables, and temporal dynamics 
% Y observed, data 

% go through points from sec 2.4 and explain how to model them, take a look at CH4 to copy/paste


% modeling exponential growth

\todo{find place for this paragraph}
The Poisson distribution arises from the law of small numbers: if there is a large population where every individual has, independently, a small probability of becoming infected in a small window of time then the total number of infections in that window of time is well approximated by the Poisson distribution.
Indeed, the law of small numbers remains valid for small dependencies \citep{Ross2011Fundamentalsa,Arratia1990Poisson}.
However, incidences observed from the SARS-CoV-2 epidemic tend to follow a negative binomial distribution \citep{Chan2021Count}. 