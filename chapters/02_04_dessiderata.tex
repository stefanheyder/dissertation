\section{Desiderata for epidemiological models}
\label{sec:dessiderata}

As the dynamics of the epidemic are constantly changing, our models should

\todo{mention problems with too simple R estimate in local outbreak setting (over / under estimation)}
% slide 3 in file:///Users/stefan/workspace/work/talks/2022-07-25%20Siegmundsburg2022%20RegionalReproductionNumbers/Heyder2022SiegmundsburgRegional.pdf
% incorporate points from spatio temprial R slide therein as well


\begin{itemize}
    \item we want models to be able to include as much data as possible, while still being numerically tractable 
\end{itemize}

\todo{this paragraph to modelling chapter}
The Poisson distribution arises from the law of small numbers: if there is a large population where every individual has, independently, a small probability of becoming infected in a small window of time then the total number of infections in that window of time is well approximated by the Poisson distribution.
Indeed, the law of small numbers remains valid for small dependencies \cite{Ross2011Fundamentalsa,Arratia1990Poisson}.
However, incidences observed from the SARS-CoV-2 epidemic tend to follow a negative binomial distribution \cite{Chan2021Count}. 

\paragraph{Regional dependencies and effects}
% use data to inform this section
\begin{itemize}
    \item German case data are reported on Landkreis level, performing analysis of each individual is not sensible 
    \item inhabitants travel between regions, and measures were taken on on regional level as well
    \item effects are not really spatial: euclidean distance is not so much of an issue but how closely connected regions are (give some examples)
    \item also want to account for other regional effects such as different socio-economic settings ... 
\end{itemize}

\paragraph{Temporal correlation}

\paragraph{Interpretability}


