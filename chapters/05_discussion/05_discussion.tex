\chapter{Discussion}
\label{cha:discussion}
\glsresetall

Let us look back on the insights gained in this thesis and put the theoretical and applied results into a broader picture. To this end, let us give answers to the questions raised in \Cref{cha:introduction} and see how far we have come. First, let us reiterate the ambitious goal of this thesis detailed in the introduction: provide mathematically sound statistical tools that allow practitioners to design and fit models suited for the epidemiological data arising in an ongoing epidemic. To this end we have contributed results at three different levels:
\begin{itemize}
    \item \textbf{applied:} the results of concrete models applied to \gls{c19} in Germany in \Cref{cha:analysis_of_selected_models},
    \item \textbf{methodological:} the general modeling strategy of \glspl{pgssm} for epidemiological models in \Cref{cha:analysis_of_selected_models}, as well as applying the \gls{cem} to \glspl{ssm} in \Cref{subsec:markov-approach}, and
    \item \textbf{theoretical:} the rigorous mathematical study of the computational methods used to fit these models, especially the comparison of \gls{eis} and the \gls{cem} in \Cref{cha:state_space_models}.
\end{itemize}

% outbreak inherently random, need stochastic / statistical modesl
In \Cref{chap:epidemiological_considerations} we highlighted the need for explainable statistical models in infectious disease epidemiology. Throughout this thesis we use \gls{c19} as a driving example, which is owed to the fact that this thesis is a product of this particular epidemic. However, the core questions surrounding past, current and future development of cases and derived indicators is of essence for all epidemics outbreaks, including seasonal influenza and other respiratory diseases, and is a matter of ongoing research, e.g. in the \gls{ecdc}s Respiratory Diseases Forecasting Hub \citep{ECDCRespiCast}. As a solution to this challenging task, we presented \glspl{ssm}, in particular \glspl{pgssm}, as a flexible framework for modeling the high-dimensional time-series data we are faced with. The wide range of applications in \Cref{cha:analysis_of_selected_models} demonstrates the flexibility of \glspl{ssm} as a modeling tool. 

% wide-availability of data
We have seen that the widely available data surrounding the epidemic, in particular daily reports on infections and hospitalizations, can be leveraged to fit \glspl{pgssm} and the fitted models allow for straight-forward interpretation. While the fitting procedures for these models can be quite involved, the models themselves, especially the temporal dynamics, are not, and as such our analyses can be disseminated to statisticians and practitioners alike. However, the interpretability of all models presented in this thesis is hampered by the quality of data available. Indeed, large constituents of our models are purely for dealing with weekday effects, and reporting delays. 

% mathematical insight into methods
The final contribution of this thesis is a mathematically rigorous analysis of the performance of the importance sampling methods used for inference in \Cref{cha:analysis_of_selected_models}.
To the authors' knowledge, \Cref{thm:cem-clt,thm:eis-clt} gives the first joint analysis of the \gls{cem} and \gls{eis}, and sheds insight on the poor performance of the \gls{cem} in practice: under reasonable conditions, we can expect the asymptotic covariance matrix of the \gls{cem} to be larger than that of \gls{eis}, as its meat matrix $M$ is fixed, whereas that of \gls{eis} can be expected to be smaller when the optimal proposal is close to the target. 
Additionally, we have shown how to apply the \gls{cem} to \glspl{pgssm} (\Cref{subsec:markov-approach}), by exploiting the dependency structure present in the target distribution. 

Our theoretical findings in \Cref{cha:state_space_models}, which highlighted the efficiency of \gls{eis} over the \gls{cem}, provided the confidence to apply this method to the complex models in \Cref{cha:analysis_of_selected_models}, where computational performance is critical. However, for the high-dimensional model presented in \Cref{sec:regional_growth_factor_model} \gls{eis} encountered numerical problems, which lead us to use the numerically very stable \gls{la} instead. Thus research into improving the numerical stability of \gls{eis} is worthwhile.

% outlooks
Our results open up several avenues for future research, from applied to theoretical. The modeling framework of \glspl{pgssm} can be applied to other epidemics, such as seasonal influenza or other respiratory diseases, where similar data are available. The models presented in \Cref{cha:analysis_of_selected_models} can be adapted to these diseases with minor modifications, allowing for a transfer of knowledge and methods across different epidemiological contexts. Here it is crucial to adapt the models presented in this thesis to account for the specific characteristics of each disease, such as transmission dynamics and reporting practices, as we have done in \Cref{cha:analysis_of_selected_models}. 
Additionally, compiling the modeling techniques --- e.g. accounting for reporting delays, weekly aggregation, nowcasting --- into a modeling framework 
that is accessible to a broader, applied audience, will be a worthwhile effort, though beyond the scope of this thesis.

Regarding the methodological results, our application of the \gls{cem} exploits the dependency structure present in the target distribution. This approach can also be applied to targets with sparse precision matrices that occur, e.g., in spatial statistics or after performing a Veccia approximation \citep{Katzfuss2021General}, which imposes a sparse dependency structure. 

% open Qs: 
While our theoretical results on the asymptotic covariance of the \gls{cem} and \gls{eis} focus on the number of samples required for convergence, it is still unclear which of the two methods produces better approximations for any given target, e.g. in terms of bias \eqref{eq:agapiou_bias} or \gls{mse} \eqref{eq:agapiou_mse} for a specific $f$. 

Additionally, our case studies focus on the Gaussian exponential family due to its widespread use, compatibility with the \gls{pgssm} targets, and analytical tractability. Nevertheless, using Gaussian proposal may be limited, as the proposals generated are symmetric and do not allow for heavy tails. For univariate distributions, there is a wide range of exponential families to choose from, which are used for importance sampling in other fields of application. For example, the \gls{cem} is used regularly in rare-event estimation together with the exponential distribution.

% what is with practical "bootstrapping" multiple sampling with CRNs?
%% theorems do not account for this CRN approach, and it gets involved quickly
%% why it improves performance not so clear

% techniques to reduce asymptotic variance by using good proposal G
%% for CEM: G has potentially large contribution to asymptotic variance, not clear how exactly it shoudl be chosen

%% what about non-gaussian EF? can we guide choice of EF?
%% spherical distribution case!
%% inverse problems + IS?

In conclusion, our thesis provides theoretical, methodological and applied insights that support the statistical modeling of epidemiological data. We believe its results will spark fruitful research that will assist in public health responses in future epidemics.