\section[Measures of epidemic spread and severity]{Measures of epidemic spread and severity}
\label{sec:measures_of_epidemic_spread}
A key component of any epidemiological model is how the spread of the epidemic is accounted for. As argued before, an epidemic is a complex process, driven by many different factors. To make this complexity manageable we employ simplified models for the spread of cases and, depending on the assumptions made, different measures that quantify the epidemic's spread arise. Actually, we are not only interested in the spread of the epidemic but also in the speed, i.e. the change over time, with which cases proliferate, because it allows us to make predictions about future cases and thus give recommendations about whether countermeasures should be employed or lifted.

In this thesis, we will primarily focus on two of these measures: the growth factor and the reproduction number. As we will argue, these two measures come with simple interpretations and, as such, are valuable in communicating the results of our modeling efforts not only to other researchers but also to non-experts such as the public and political stakeholders.

Additionally, we will be interested in measures that capture the severity of the epidemic, i.e. the morbidities and mortalities caused by the epidemic. As these events are consequences of infection that occur after a delay, they can be recovered from incidence data. Thus modeling the spread of the epidemic serves two goals: making inferences and predictions about the cases and associated measures, as well as morbidities.

To introduce the different measures in the following, we will, for the moment, make some simplifying assumptions about the population in which the epidemic spreads and the time frame considered. Consecutively, we will relax these assumptions to accommodate more realistic populations. 

% homogenous population
% homogenous mixing
First of all, we consider a homogenous population with homogenous mixing. This means that any two individuals in the population are affected by the epidemic in the same way: the probabilities of becoming infected, infectious, hospitalized or recovering from infection are the same for every individual in the population. Additionally, homogenous mixing indicates that once an individual is infected, they meet and infect every other individual in the population with the same probability. 
% large / infinite population, i.e. prob. of meeting / infectin twice is small
Furthermore, we assume that the population is large enough that the probability of duplicate infections, i.e. becoming infected twice either from the same or different individuals, is negligibly small. Similarly, we could also assume that the population is infinitely large or that the time frame under consideration is sufficiently short.
% constant behavior
Finally, we assume that the behavior of the population is constant over the period modeled. 

We will 

\subsection{Growth Factor}
\label{subsec:growth_factor}

\subsection{Reproduction number}
\label{subsec:reproduction_number}

\subsection{Measures of epidemic severity}
\label{subsec:measures-severity}

\subsection{Other indicators}
\label{subsec:other_indicat}

\subsection{Usefulness of indicators}
\label{subsec:usefulness_of_indicators}

