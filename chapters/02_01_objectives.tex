\section{Objectives of epidemiological modelling}

\paragraph{Monitoring}
\begin{itemize}
    \item monitoring is real-time scenario, interested in current developments, i.e. recent past and near future. complicated by potentially slow reporting, data revisions
    \item informs decision makers on whether measures should be taken
    \item ForecastHub(s) provide platform that creates ensemble forecast to obtain better predictions \cite{Bracher2022National,Bracher2021Preregistered,Ray2020Ensemble,Sherratt2022Predictive}
\end{itemize}
\paragraph{Retrospective Analysis}
\begin{itemize}
    \item evaluation of measures taken, want interpretation as causal as possible
    \item informs decision makes on which measures were effective and how much
    \item difficult due to usual reasons: poor data quality, observational data, causual structure difficult, early/late adoption makes timing of measurements difficult
    \item cite some papers that did this \cite{Flaxman2020Estimating,Brauner2021Inferring,Khazaei2023Using}
\end{itemize}
\paragraph{Scenario Modelling}

\begin{itemize}
    \item concerns itself with modelling the impact that variants, seasonality etc. have in specific scenarios 
    \item find out whether there is already paper of ECDC to cite
\end{itemize}
